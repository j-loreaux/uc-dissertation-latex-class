% arara: clean: {files: [dissertation.aux, dissertation.bbl, dissertation.bcf, dissertation.blg, dissertation.log, dissertation.out, dissertation.run.xml, dissertation.synctex.gz, dissertation.toc, introduction.aux, background.aux, results.aux, prerequisite-material.aux]}
% arara: pdflatex 
% arara: biber
% arara: pdflatex
% arara: pdflatex
% bash: rm dissertation.[bgilors]* dissertation.toc *\~ *.aux 
\documentclass[chap,comply,bibliography=totoc]{uc-dissertation} 


%%%%% Essential packages

% Basic math packages
\usepackage{amsmath}
\usepackage{amssymb}
\usepackage{amsthm}
% For making references
\usepackage[style=alphabetic,firstinits=true,maxnames=10]{biblatex}



%%%%% Packages for the electronic version

% Create hyperlinks for cross references: load last unless otherwise specified
\usepackage[hidelinks]{hyperref}




%%%%% New theorem-like environments

%%% amsthm environments
% number the 'theorem' environments by section
\newtheorem{theorem}{Theorem}[section]
% make all other environments use the 'theorem' counter
% so as to be consistent. This will produce things like
% Theorem 1.1, Definition 1.2, Lemma 1.3, Theorem 1.4, etc. 
\newtheorem{question}[theorem]{Question}
\newtheorem{problem}[theorem]{Problem}
\newtheorem{lemma}[theorem]{Lemma}
\newtheorem{corollary}[theorem]{Corollary}
\newtheorem{proposition}[theorem]{Proposition}
\newtheorem{conjecture}[theorem]{Conjecture}
\theoremstyle{definition}
\newtheorem{definition}[theorem]{Definition}
\newtheorem{remark}[theorem]{Remark}
\newtheorem{example}[theorem]{Example}
% Equation numbering by section to match theorem numbering
\numberwithin{equation}{section}



%%%%% Input front matter material

%%%%% Set titlepage information

%%% All desired titlepage info must be set before maketitle is called.

\title{This is a title that \\ has multiple lines} 
\subtitle{This subtitle is fairly short}

%% The next command is a redefined date command that takes two
%% arguments of the form {month}{year}, where 'month' is the 
%% two-digit month, and 'year' is the four-digit year. In order 
%% to activate this on the title page you must pass the option 
%% 'fixed' to \maketitle, otherwise it will use \today
\date{05}{2015}
\author{Lowly Candidate}
%% The next command takes 3 mandatory and 1 optional argument.
%% The optional argument is the number of previous degrees held
%% by the candidate and defaults to 1. The mandatory arguments
%% are semicolon separated lists of, respectively, the degrees 
%% held, years obtained, and schools conferring those degrees.
\previousdegrees[3]{%
  B.A. Chemistry; M.S. Physics; M.S. Mathematics }{%
  2000; 2002; 2007 }{%
  University of Somewhere; University of Nowhere; University of Nebraska %
}

\chair{Random Advisor}
%% The next command takes 1 mandatory and 1 optional argument.
%% The optional argument is the number of committee members.
%% defaults to 1. The mandatory argument is a semicolon 
%% separated list of names of the committee members.
\committee[3]{Other Member; Another Member; Y.A. Member}


%% The following commands default to 'Mathematical Sciences',
%% 'McMicken College of Arts and Sciences', and 'University 
%% of Cincinnati' if they are not explicitly set by the user.
\department{Mathematical Sciences}
\college{Eponymous College of Arts and Sciences}
\university{University of Somewhere}

%%% Local Variables: 
%%% mode: latex
%%% TeX-master: "dissertation"
%%% End: 

\abstract{%
This is the abstract.
}

%%% Local Variables: 
%%% mode: latex
%%% TeX-master: "dissertation"
%%% End: 

\acknowledgments{%
This is where we thank people and recognize funding sources.
}

%%%%% Set bibliography file(s)

\addbibresource{references.bib}


\begin{document} 


%%%%% This starts the beginning material of the document. It resets the page
%%%%% numbering to start at 1 and to display as roman numerals.
\frontmatter

%%%%% This typesets the title page as defined by the uc-dissertation class.
%%%%% It takes an optional argument which, if set to 'fixed', displays the
%%%%% month and year for the date defined above by \newdate instead of \today.
\maketitle      %% or \maketitle[fixed]

%% This command typesets the abstract defined by the user with the \abstract
%% command. It starts a new page (on the right if the document is two-sided),
%% and adds the abstract to the Table of Contents (really it typesets a 
%% chapter).
\abstractpage

%% This makes the copyright page. It takes an optional argument which allows 
%% the user to choose from multiple copyrights. The default is 'all': the 
%% author reserves all rights to the document. This may be necessary depending 
%% on journal restrictions. However, by setting the optional argument to empty, 
%% i.e. [], the user can obtain a blank page instead of a copyright page (one
%% of these two things is required by UC. Also, by using the strings 'cc',
%% 'by','nc','nd', and 'sa', the user may choose a Creative Commons license.
%% The only two accepted by UC are 'cc-by-nc-nd' and 'cc-by-nc-sa'. Note that
%% the dashes are not strictly necessary and order doesn't matter. It is up to
%% the user to come up with a valid Creative Commons license.
%%
%% TODO: add a 'custom' option which allows the user to build the copyright 
%% page from scratch.
\copyrightpage
%% The above is equivalent to \copyrightpage[all]
%% \copyrightpage[]
%% \copyrightpage[cc-by-nc-nd]
%% \copyrightpage[cc-by-nc-sa]


%% This command typesets the acknowledgments defined by the user with the 
%% \acknowledgments command. It starts a new page (on the right if the document 
%% is two-sided), and adds the acknowledgments to the Table of Contents (really 
%% it typesets a chapter).
\acknowledgmentspage

%% Typeset the table of contents
\tableofcontents 


%%%%% This starts page numbering anew, with arabic numerals as opposed to the roman
%%%%% numerals of the front matter.
\mainmatter 

\chapter{Introduction}
\label{cha:introduction}
\chapterquote{Chapter quotes can bring an air of sophistication to your work. Just be sure they aren't signed `anonymous'.}{Anonymous}

\section{Gentle beginnings}

\subsection{The art of introduction}

This is where you give the first \index{impression} impression of your work. 
It is important that it comes across as interesting and worthy of study. 
Make sure you state the major problems you will solve later.
Note that now the material is double-spaced instead of single-spaced (assuming you are using the \verb+comply+ class option).
We will insert a short table so that one appears in the List of Tables.

\begin{table}[h]
  \centering
  \begin{tabular}{c|c}
    $x$ & $y$ \\
    This & That \\
    These & Those \\
    Here & There \\
  \end{tabular}
  \caption{A table to supercede all other tables}
  \label{tab:supercession}
\end{table}

\newpage

\subsection{Harsh second dates}

Unfortunately, not all is well-and-good in the land of make believe dissertations.



%%% Local Variables: 
%%% mode: latex
%%% TeX-master: "dissertation"
%%% End: 


\chapter{Background}

In this chapter we review the \gls{literature} relevant to our work. 
Here you most likely should have some references \cite{Tol-JoBN-1957}. 

%%% Local Variables: 
%%% mode: latex
%%% TeX-master: "dissertation"
%%% End: 


\chapter{Results}

\chapterquote{God created the integers, all else is the work of man.}{Leopold Kronecker}

This is the \index{chapter} chapter where we prove our main results. This will probably take more than one chapter. Suppose that God created the \index{positive} positive integers $\mathbb{N}$ and nothing else, then we can conclude...

%%% Local Variables: 
%%% mode: latex
%%% TeX-master: "dissertation"
%%% End: 


%%%%% This marks the end of the primary content and signals the start of additional
%%%%% content like the bibliography, glossary and index.
\backmatter

%% Bibliography
\printbibliography

%%%%% We have to use mainmatter again for spacing issues and for the correct 
%%%%% labeling of appendix chapter in the ToC because UC has bad page order.
%%%%% Below we use the mainmattertwo command defined by the uc-dissertation
%%%%% class so that the page order is not reset.
\mainmattertwo

%% Appendices
\appendix

\chapter{Prerequisite material}

We need a figure somewhere, so let's put it here.

\begin{figure}[h]
  \centering
  I'm a float floating away. Float$\ldots$ float$\ldots$ float$\ldots$
  \caption{Hi, I'm a floating figure!}
  \label{fig:floating-stuff}
\end{figure}

%%% Local Variables: 
%%% mode: latex
%%% TeX-master: "dissertation"
%%% End: 



\end{document}
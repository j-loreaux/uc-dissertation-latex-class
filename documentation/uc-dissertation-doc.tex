\documentclass{article}

\usepackage{xcolor}
\usepackage{graphicx}
\usepackage{listings}
\usepackage{multicol}
\usepackage{hyperref}

\title{The \texttt{uc-dissertation} document class}
\author{Jireh Loreaux}

\begin{document}

\definecolor{darkgreen}{rgb}{0,0.75,0}
\lstdefinestyle{tex-command-syntax}{%
  language={[LaTeX]TeX}, 
  basicstyle=\ttfamily\upshape,
  texcsstyle=*\color{blue},
  xleftmargin=\parindent,
  commentstyle=\color{gray},
  moredelim=[is][\color{orange}\rmfamily\itshape]{+}{+},
  moredelim=[is][\color{darkgreen}\rmfamily\itshape]{|}{|},
  literate={>}{{$\rangle$}}{1}
  {<}{{$\langle$}}{1}
  }
\lstset{style=tex-command-syntax}

\maketitle

\section{Purpose and requirements}

The purpose of this \LaTeX{} class is to provide an all-inclusive user-friendly method for students (especially in mathematics) at the University of Cincinnati to write their dissertations. 
The University of Cincinnati has some very strict guidelines for preparing the dissertation.
The Electronic Thesis and Dissertation (\href{http://grad.uc.edu/student-life/etd.html}{ETD}) page has a timeline for completion during the last semester and the various steps along the way.
There is also a \href{http://grad.uc.edu/student-life/etd/formatting.html}{formatting} page that lists a series of requirements, which we reproduce below.

\begin{description}
\item[Spacing] Double-space all text. Long quotations and footnotes may be single-spaced.
\item[Font] 11-point or larger is recommended for readability. 
No matter what font you choose, all fonts \emph{must} be embedded in your final PDF.
\item[Margins] Approximately one inch of white space should be at the top, bottom and sides on each page.
\item[Page Numbers] Except for the title page, number all pages in your ETD.
  \begin{itemize}
  \item Number the pages preceding the body of the text with small
    roman numerals (i, ii, iii, iv, etc.), placed at the bottom center
    of the page.  Remember, the page number for the Title Page (i)
    should not be visible.  
  \item Number all pages through the body,
    bibliography, appendices and index with Arabic numerals (1, 2, 3,
    etc.).  All numbering should be in the bottom-center of the page.
  \end{itemize}
\item[Footnotes] Recommended placement of footnotes is at the bottom of the appropriate page. 
  It is not advisable to place them at the end of chapters where they are difficult to consult. 
  Please consult with your committee to determine which style is appropriate for your field.
\item[Illustrations] May consist of line drawings, graphs, maps, photographs, chemical formulas, or musical scores or passages. 
  All of these should be inserted into the PDF document (not linked to external sources).
\item[Title] Do \emph{not} use all capital letters for your title. 
  Capital letters are still appropriate for acronyms, proper nouns, first letter, etc.
\end{description}

The \href{http://grad.uc.edu/student-life/etd/page_order.html}{Required Page Order} is as follows:

\begin{multicols}{2}
  \begin{enumerate}
  \item Title page
  \item Abstract
  \item Copyright notice
  \item Acknowledgments (O)
  \item Table of Contents
  \item List of Tables (IF)
  \item List of Figures (IF)
  \item List of Symbols (IF)
  \item Body text
  \item Glossary (IF)
  \item Bibliography
  \item Appendices (IF)
  \item Index (IF)
  \item Audio/Video
  \end{enumerate}
\end{multicols}

Those items followed by (IF) signify that they should only be included \emph{if needed}. 
For example, if a student has only one figure throughout his/her entire dissertation, there is no reason to include a List of Figures.
If, on the other hand, the student has over twenty figures, it would probably be good practice to include such a list. 
Use your best judgment for each item. The Acknowledgments page is optional, as indicated by the (O) appended to its entry. 
If a student desires a Dedication in addition to, or instead of, the Acknowledgments, it should be placed either immediately preceding or immediately succeeding the acknowledgments.

There are a few special points that need to be addressed. 
First and foremost, the abstract must be \emph{500 words or less}, with \emph{no exceptions}. 
Including tables and figures in the abstract is strongly discouraged. 
Please check with the Graduate School if you must use tables or figures in your abstract. 
While the graduate school doesn't specify about the use of symbols (mathematical or otherwise) in the abstract, it is generally poor practice to include such in an abstract. 
The reason for this is that abstracts are often posted in places where the typesetting is somewhat simplistic and does not easily accommodate the inclusion of symbols. 
Similarly, because the abstract should be able to be dissociated from the complete dissertation, one should not insert citations using the bibliography in the abstract; the authors last names will suffice for this purpose if necessary. 

Another small point is that if the student plans to include video, only MPEG, MP4, and AVI files can be inserted directly into the PDF.
Finally, there are several items which are required to appear on the title page. These are:
\begin{multicols}{2}
  \begin{enumerate}\itemsep0pt
  \item obviously, the title,
  \item author name and date,
  \item previous degrees,
  \item degree to be conferred,
  \item department and college, and
  \item name of committee chair
  \end{enumerate}
\end{multicols}

\section{Provisions and user input}

\subsection{Class base and options}

The \verb+uc-dissertation+ class is based on the KOMA-script book class \verb+scrbook+. 
As a result, you may pass any options of the \verb+scrbook+ class; however, those relating to elements redefined by \verb+uc-dissertation+ may not work as expected, especially those pertaining to the title page. 


\begin{lstlisting}
\documentclass[+<mand args>+]{|<classname>|}
\end{lstlisting}

\begin{figure}
  \fbox{%
    \includegraphics[width=\linewidth]{titlepage-commands.pdf}
  }
  \caption{User input and the title page}
\end{figure}

\end{document}
